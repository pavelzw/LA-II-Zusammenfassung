\documentclass[main.tex]{subfiles}
\begin{document}

\section*{Multilineare Algebra}
\subsection*{Das Tensorprodukt}

\begin{karte}{Tensorprodukt}
    Sei \( K \) ein Körper und \( V,W \) Vektorräume über 
    \( K \). Dann gibt es einen \(K\)-Vektorraum \( V \otimes_K W \) 
    und eine bilineare Abbildung \( V \times W \rightarrow V\otimes_K W \) 
    mit folgender universeller Eigenschaft:

    Für jeden \( K \)-Vektorraum \( Z \) und jede bilineare Abbildung 
    \( \abb{f}{V\times W}{Z} \) gibt es genau eine lineare Abbildung 
    \( \abb{f_\otimes}{V\otimes_K W}{Z} \) so, dass das folgende Diagramm kommutiert:
    \begin{center}
        \begin{tikzcd}
            V \times W \arrow{rd}{f} \arrow{d}{\otimes} \\
            V \otimes_K W \arrow{r}{f_\otimes} & Z
        \end{tikzcd}
    \end{center}
    Der \( K \)-Vektorraum \( V \otimes_K W \) und die 
    Abbildung \( \abb{\otimes}{V\times W}{V \otimes_K W} \) 
    sind durch diese universelle Eigenschaft bis auf 
    kanonische Isomorphie eindeutig bestimmt. Man 
    bezeichnet \( V \otimes_K W \) als das \textit{Tensorprodukt} 
    von \( V \) und \(W\) über \(K\).
\end{karte}

\begin{karte}{Tensor Vektorräume}
    Die folgenden Abbildungen sind wohldefinierte zueinander inverse 
    Isomorphismen von \(K\)-Vektorräumen:
    \[ K\otimes_K V \rightarrow V, \lambda \otimes v \mapsto \lambda \cdot v \]
    \[ V \rightarrow K \otimes_K V, v \mapsto 1 \otimes v \]
\end{karte}

\begin{karte}{Tensorabbildung}
    Sind \( \abb{f}{V}{V'} \) und \( \abb{g}{W}{W'} \) 
    lineare Abbildungen, so definiert 
    \[ \abb{ f \otimes_K g }{V \otimes_K W}{V' \otimes_K W'}, 
    v \otimes w \mapsto f(v) \otimes g(w) \]
    eine wohldefinierte lineare Abbildung.\\
    Dabei gilt \( \id_V \otimes_K \id_W = \id_{V \otimes_K W} \) 
    und diese Konstruktion ist mit der Komposition von Abbildungen 
    verträglich.
\end{karte}

\begin{karte}{Rechenregeln Tensorprodukt}
    Das Tensorprodukt erfüllt folgende kanonische Isomorphismen:
    \begin{description}
        \item[Kommutativität] 
        \[ U \otimes_K V \overset{\cong}{\longrightarrow} V \otimes_K U, \quad
        u \otimes v \mapsto v\otimes u. \]
        \item[Assoziativität]
        \[ U \otimes_K (V \otimes_K W) \overset{\cong}{\longrightarrow} 
        ( U \otimes_K V ) \otimes_K W, \quad 
        u \otimes (v\otimes w) \mapsto (u\otimes v) \otimes w. \]
        \item[Verträglichkeit mit direkten Summen]
        \[ V \otimes_K \bigoplus_{i\in I} W_i \overset{\cong}{\longrightarrow} 
        \bigotimes_{i \in I} (V \otimes_K W_i), \quad 
        v \otimes \sum w_i \mapsto \sum v \otimes w_i. \] 
        \item[Exponentialgesetz]
        \[ \hom_K(V \otimes_K W, U) \overset{\cong}{\longrightarrow} 
        \hom_K(W, \hom_K(V, U)), \]
        \[ f \mapsto \left(w \mapsto (v\mapsto f(v\otimes w))\right). \]
    \end{description}
\end{karte}

\begin{karte}{Dimension Tensorräume}
    Es seien \(V\) und \(W\) endlich-dimensionale \(K\)-Vektorräume. 
    Dann gilt:
    \[ \dim_K (V \otimes_K W) = \dim_K(V) \cdot \dim_K(W). \]
\end{karte}

\begin{karte}{Isomorphismus Dualräume}
    Es seien \(V\) und \(W\) endlich-dimensionale \(K\)-Vektorräume. Die 
    Abbildung 
    \[ V^* \otimes_K W \rightarrow \hom_K(V,W), 
    \phi \otimes w \mapsto (v\mapsto \phi(v) \cdot w) \]
    ist ein (wohldefinierter) Isomorphismus.
\end{karte}

\subsection*{Anwendungen Tensorprodukt}

\begin{karte}{Koordinatenfreie Beschreibung der Spur}
    Sei \(V\) endlich-dimensional. Sei \( \abb{\psi}{\hom_K(V,V)}{V^* \otimes V} \) 
    das Inverse des Isomorphismus 
    \[ \phi:V^* \otimes_K V\overset{\cong}{\longrightarrow} \hom_K(V,V), \quad 
    \phi \otimes v \mapsto (x \mapsto \phi(x) \cdot v). \]
    Sei 
    \[ \abb{E}{V^* \otimes_K V}{K}, \quad \phi \otimes v \mapsto \phi(v) \] 
    die Auswertungsabbildung. Dann gilt für jeden Endomorphismus 
    \( \abb{f}{V}{V} \) 
    \[ \Spur f = E \circ \psi(f). \]
\end{karte}

\end{document}
