\documentclass[main.tex]{subfiles}
\begin{document}

\section*{Klassifikation Endomorphismen}
\subsection*{Invariante Unterräume}

\begin{karte}{Definition Algebra}
    Eine \(K\)-Algebra besteht aus einem Ring \(A\) 
    und einer Abbildung 
    \[ \abb{*}{K\times A}{A} \]
    mit folgenden Eigenschaften:
    \begin{itemize}
        \item Die additive Gruppe von \(A\) bildet 
        zusammen mit \( * \) einen \(K\)-Vektorraum.
        \item Die Skalarmultiplikation \(*\) ist mit der 
        Ringmultiplikation verträglich:
        \[ \alpha * (x \cdot y) = (\alpha * x) \cdot y 
        = x \cdot (\alpha * y). \]
    \end{itemize}
    Eine Abbildung zwischen \(K\)-Algebren ist ein 
    (\(K\)-)Algebrenhomomorphismus, wenn sie gleichzeitig ein Ring- 
    und ein \(K\)-Vektorraumhomomorphismus ist.
\end{karte}

\begin{karte}{Universelle Eigenschaft des Polynomrings}
    Ist \(A\) eine \(K\)-Algebra und \( a\in A \), so gibt 
    es genau einen \\
    \(K\)-Algebrenhomomorphismus 
    \[ \abb{\phi}{K[X]}{A} \text{ (Einsetzungshomomorphismus)} \] 
    mit 
    \[ \phi(X) = a. \]
    Ist \( p\in K[X] \), so schreibt man auch \( p(a) := \phi(p) \).\\
    Sei \( \abb{f}{V}{V} \) ein Endomorphismus und \( p\in K[X] \). 
    Dann ist \( \ker p(f) \) ein \( f \)-invarianter Untervektorraum.
\end{karte}

\begin{karte}{\( f \)-invariant, \(f\)-zyklisch}
    Sei \( \abb{f}{V}{V} \) ein Endomorphismus. Ein Untervektorraum 
    \( W\subset V \) heißt 
    \begin{enumerate}
        \item \( f \)-invariant, wenn \( f(W) \subset W \).
        \item \(f\)-zyklisch, wenn 
        \( \langle \set{ w, f(w), f^2(w),\ldots }\rangle = W \) für 
        ein \( w\in W \).
    \end{enumerate}
\end{karte}

\begin{karte}{Satz von Cayley-Hamilton}
    FÜr einen Endomorphismus \( \abb{f}{V}{V} \) eines endlich-dimensionalen 
    \( K \)-Vektorraums gilt 
    \[ \chi_f(f) = 0. \]
\end{karte}

\begin{karte}{Teilerfremdheit}
    Sind \( p, p', r\in K[X] \) Polynome mit \( p = r \cdot p' \), 
    so nennt man \(r\) \textit{Teiler} von \(p\). Zwei Polynome \(p,q\in K[X]\) heißen 
    \textit{teilerfremd}, falls jeder gemeinsame Teiler konstant ist.\\
    Sind \( p, q\in K[X] \) teilerfremd, so gibt es Polynome \( a,b\in K[X] \) 
    mit \( a\cdot p + b \cdot q = 1 \in K[X] \).\\
    Zwefällt ein Polynom in Linearfaktoren, so auch jeder Teiler des Polynoms.\\
    Seien \( \lambda, \mu \in K \) verschieden. Seien \(n,m\in \N\). Dann sind die 
    Polynome \( (X - \lambda)^n \) und \( (X - \mu)^m \) teilerfremd.
\end{karte}

\begin{karte}{Teilerfremde Polynome und deren Kerne}
    Sind \( p,q\in K[X] \) teilerfremde Polynome \( \abb{f}{V}{V} \) 
    ein \(K\)-linearer Endomorphismus, dann gilt 
    \[ \ker(p\cdot q)(f) = \ker p(f) \oplus \ker q(f). \]
    Seien \( \lambda_1, \ldots, \lambda_r \) paarweise verschieden und 
    \( n_1,\ldots,n_r \in \N \). Seien \( p_i(X) = (X - \lambda_i)^{n_i} 
    \in K[X] \) und \( p(X) \in K[X] \) das Produkt der \( p_1(X), \ldots,p_r(X) \). \\
    Weiter sei \( f\in \hom(V,V) \) ein Endomorphismus. Dann gilt 
    \[ \ker p(f) = \ker p_1(f) \oplus \cdots \oplus \ker p_r(f). \]
\end{karte}

\begin{karte}{Direkte Summe von Abbildungen}
    Seien \( \abb{f_i}{U_i}{V_i}, i \in \set{1,\ldots,k} \), 
    \(K\)-lineare Abbildungen. Deren \textit{direkte Summe} ist die durch 
    \[ \abb{ \bigoplus_{i=1}^k f_i }{ \bigoplus_{i=1}^k U_i }{ \bigoplus_{i=1}^k V_i }, \quad
    (u_1,\ldots, u_k) \mapsto (f_1(u_1), \ldots, f_k(u_k)) \]
    definierte \(K\)-lineare Abbildung.\\
    Für die direkte Summe der \(K\)-linearen Abbildungen 
    \( \abb{f_i}{U_i}{V_i}, i \in \set{1,\ldots,k} \) gilt 
    \[ \ker \bigoplus_{i=1}^k f_i = \bigoplus_{i=1}^k \ker f_i. \]
    Insbesondere gilt 
    \( \dim \ker \bigoplus_{i=1}^k f_i = \sum_{i=1}^k \dim \ker(f_i) \).
\end{karte}

\begin{karte}{Blockmatrix}
    Seien \( A_i \in M_{n_i}(K) \) für \( i\in \set{1,\ldots,k} \). 
    Wir bezeichnen durch 
    \[ A_1 \boxplus A_2 \boxplus \cdots \boxplus A_k \] 
    diejenigen Matrix in \( M_n(K) \) mit \( n=n_1 + \cdots + n_k \), 
    deren Diagonalblöcke die Matrizen \( A_i \) sind und deren restliche 
    Einträge Null sind. Sie wird als Blocksumme der Matrizen 
    \( A_1, \ldots, A_k \) bezeichnet.
\end{karte}

\begin{karte}{Direkte Summe von Endomorphismen als Matrix}
    Sei \( f \) die direkte Summe der Endomorphismen 
    \( \abb{f_i}{V_i}{V_i}, i\in \set{1,\ldots, k} \) 
    mit \( \dim V_i = n_i < \infty \). Sei \( B_i \) 
    eine Basis von \( V_i \). Sei \(B\) die Basis von 
    \( V_1 \oplus \cdots \oplus V_k \), die durch 
    Konkatenation von \( B_1, \ldots, B_k \) entsteht. 
    Dann gilt 
    \[ M_{BB}(f) = M_{B_1B_1}(f_1) \boxplus \cdots \boxplus M_{B_k B_k}(f_k)
    \in M_{ n_1 + \cdot n_k }(K). \]
    Es gilt außerdem \( \det (A_1 \boxplus \cdots A_k) = \det A_1 \cdots \det A_k \).
\end{karte}

\end{document}