\documentclass[main.tex]{subfiles}
\begin{document}

\section*{Test}
\subsection*{Test}
\begin{karte}{Definition Algebra}
    Eine \(K\)-Algebra besteht aus einem Ring \(A\) 
    und einer Abbildung 
    \[ \abb{*}{K\times A}{A} \]
    mit folgenden Eigenschaften:
    \begin{itemize}
        \item Die additive Gruppe von \(A\) bildet 
        zusammen mit \( * \) einen \(K\)-Vektorraum.
        \item Die Skalarmultiplikation \(*\) ist mit der 
        Ringmultiplikation verträglich:
        \[ \alpha * (x \cdot y) = (\alpha * x) \cdot y 
        = x \cdot (\alpha * y). \]
    \end{itemize}
    Eine Abbildung zwischen \(K\)-Algebren ist ein 
    (\(K\)-)Algebrenhomomorphismus, wenn sie gleichzeitig ein Ring- 
    und ein \(K\)-Vektorraumhomomorphismus ist.
\end{karte}
\begin{karte}{Universelle Eigenschaft des Polynomrings}
    Ist \(A\) eine \(K\)-Algebra und \( a\in A \), so gibt 
    es genau einen \\
    \(K\)-Algebrenhomomorphismus 
    \[ \abb{\phi}{K[X]}{A} \text{ (Einsetzungshomomorphismus)} \] 
    mit 
    \[ \phi(X) = a. \]
    Ist \( p\in K[X] \), so schreibt man auch \( p(a) := \phi(p) \).\\
\end{karte}

\end{document}