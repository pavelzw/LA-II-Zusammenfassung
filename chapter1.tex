\documentclass[main.tex]{subfiles}
\begin{document}

\section*{Klassifikation Endomorphismen}
\subsection*{Invariante Unterräume}

\begin{karte}{Definition Algebra}
    Eine \(K\)-Algebra besteht aus einem Ring \(A\) 
    und einer Abbildung 
    \[ \abb{*}{K\times A}{A} \]
    mit folgenden Eigenschaften:
    \begin{itemize}
        \item Die additive Gruppe von \(A\) bildet 
        zusammen mit \( * \) einen \(K\)-Vektorraum.
        \item Die Skalarmultiplikation \(*\) ist mit der 
        Ringmultiplikation verträglich:
        \[ \alpha * (x \cdot y) = (\alpha * x) \cdot y 
        = x \cdot (\alpha * y). \]
    \end{itemize}
    Eine Abbildung zwischen \(K\)-Algebren ist ein 
    (\(K\)-)Algebrenhomomorphismus, wenn sie gleichzeitig ein Ring- 
    und ein \(K\)-Vektorraumhomomorphismus ist.
\end{karte}

\begin{karte}{Universelle Eigenschaft des Polynomrings}
    Ist \(A\) eine \(K\)-Algebra und \( a\in A \), so gibt 
    es genau einen \\
    \(K\)-Algebrenhomomorphismus 
    \[ \abb{\phi}{K[X]}{A} \text{ (Einsetzungshomomorphismus)} \] 
    mit 
    \[ \phi(X) = a. \]
    Ist \( p\in K[X] \), so schreibt man auch \( p(a) := \phi(p) \).\\
    Sei \( \abb{f}{V}{V} \) ein Endomorphismus und \( p\in K[X] \). 
    Dann ist \( \ker p(f) \) ein \( f \)-invarianter Untervektorraum.
\end{karte}

\begin{karte}{\( f \)-invariant, \(f\)-zyklisch}
    Sei \( \abb{f}{V}{V} \) ein Endomorphismus. Ein Untervektorraum 
    \( W\subset V \) heißt 
    \begin{enumerate}
        \item \( f \)-invariant, wenn \( f(W) \subset W \).
        \item \(f\)-zyklisch, wenn 
        \( \langle \set{ w, f(w), f^2(w),\ldots }\rangle = W \) für 
        ein \( w\in W \).
    \end{enumerate}
\end{karte}

\begin{karte}{Satz von Cayley-Hamilton}
    FÜr einen Endomorphismus \( \abb{f}{V}{V} \) eines endlich-dimensionalen 
    \( K \)-Vektorraums gilt 
    \[ \chi_f(f) = 0. \]
\end{karte}

\begin{karte}{Teilerfremdheit}
    Sind \( p, p', r\in K[X] \) Polynome mit \( p = r \cdot p' \), 
    so nennt man \(r\) \textit{Teiler} von \(p\). Zwei Polynome \(p,q\in K[X]\) heißen 
    \textit{teilerfremd}, falls jeder gemeinsame Teiler konstant ist.\\
    Sind \( p, q\in K[X] \) teilerfremd, so gibt es Polynome \( a,b\in K[X] \) 
    mit \( a\cdot p + b \cdot q = 1 \in K[X] \).\\
    Zwefällt ein Polynom in Linearfaktoren, so auch jeder Teiler des Polynoms.\\
    Seien \( \lambda, \mu \in K \) verschieden. Seien \(n,m\in \N\). Dann sind die 
    Polynome \( (X - \lambda)^n \) und \( (X - \mu)^m \) teilerfremd.
\end{karte}

\begin{karte}{Teilerfremde Polynome und deren Kerne}
    Sind \( p,q\in K[X] \) teilerfremde Polynome \( \abb{f}{V}{V} \) 
    ein \(K\)-linearer Endomorphismus, dann gilt 
    \[ \ker(p\cdot q)(f) = \ker p(f) \oplus \ker q(f). \]
    Seien \( \lambda_1, \ldots, \lambda_r \) paarweise verschieden und 
    \( n_1,\ldots,n_r \in \N \). Seien \( p_i(X) = (X - \lambda_i)^{n_i} 
    \in K[X] \) und \( p(X) \in K[X] \) das Produkt der \( p_1(X), \ldots,p_r(X) \). \\
    Weiter sei \( f\in \hom(V,V) \) ein Endomorphismus. Dann gilt 
    \[ \ker p(f) = \ker p_1(f) \oplus \cdots \oplus \ker p_r(f). \]
\end{karte}

\begin{karte}{Direkte Summe von Abbildungen}
    Seien \( \abb{f_i}{U_i}{V_i}, i \in \set{1,\ldots,k} \), 
    \(K\)-lineare Abbildungen. Deren \textit{direkte Summe} ist die durch 
    \[ \abb{ \bigoplus_{i=1}^k f_i }{ \bigoplus_{i=1}^k U_i }{ \bigoplus_{i=1}^k V_i }, \quad
    (u_1,\ldots, u_k) \mapsto (f_1(u_1), \ldots, f_k(u_k)) \]
    definierte \(K\)-lineare Abbildung.\\
    Für die direkte Summe der \(K\)-linearen Abbildungen 
    \( \abb{f_i}{U_i}{V_i}, i \in \set{1,\ldots,k} \) gilt 
    \[ \ker \bigoplus_{i=1}^k f_i = \bigoplus_{i=1}^k \ker f_i. \]
    Insbesondere gilt 
    \( \dim \ker \bigoplus_{i=1}^k f_i = \sum_{i=1}^k \dim \ker(f_i) \).
\end{karte}

\begin{karte}{Blockmatrix}
    Seien \( A_i \in M_{n_i}(K) \) für \( i\in \set{1,\ldots,k} \). 
    Wir bezeichnen durch 
    \[ A_1 \boxplus A_2 \boxplus \cdots \boxplus A_k \] 
    diejenigen Matrix in \( M_n(K) \) mit \( n=n_1 + \cdots + n_k \), 
    deren Diagonalblöcke die Matrizen \( A_i \) sind und deren restliche 
    Einträge Null sind. Sie wird als Blocksumme der Matrizen 
    \( A_1, \ldots, A_k \) bezeichnet.
\end{karte}

\begin{karte}{Direkte Summe von Endomorphismen als Matrix}
    Sei \( f \) die direkte Summe der Endomorphismen 
    \( \abb{f_i}{V_i}{V_i}, i\in \set{1,\ldots, k} \) 
    mit \( \dim V_i = n_i < \infty \). Sei \( B_i \) 
    eine Basis von \( V_i \). Sei \(B\) die Basis von 
    \( V_1 \oplus \cdots \oplus V_k \), die durch 
    Konkatenation von \( B_1, \ldots, B_k \) entsteht. 
    Dann gilt 
    \[ M_{BB}(f) = M_{B_1B_1}(f_1) \boxplus \cdots \boxplus M_{B_k B_k}(f_k)
    \in M_{ n_1 + \cdot n_k }(K). \]
    Es gilt außerdem \( \det (A_1 \boxplus \cdots \boxplus A_k) = \det A_1 \cdots \det A_k \).
\end{karte}

\subsection*{Jordan-Normalform}

\begin{karte}{\(\lambda\)-Jordankästchen}
    Die Matrix \(J_n(\lambda)\) heißt \(\lambda\)- Jordankästchen der
    Länge \(n\).
    \[J_4(\lambda)\begin{pmatrix}
        \lambda &0 &0 &0\\
        1 &\lambda & 0 &0\\
        0 &1 &\lambda &0\\
        0 &0 &1 &\lambda
    \end{pmatrix}\]
    \(\rk J_n(0) = n-1\)\\
    \(J_n(0)\) wird auch als Rechtsverschiebung bezeichnet. \(J_n(\lambda)\)
    als erweiterte \(\lambda\)-Rechtsverschiebung.\\
    Ein Jordankästchen hat nur den Eigenwert \(\lambda\) mit algebraischer Vielfachheit
    von \(n\) und geometrischer Vielfachheit \(1\).
\end{karte}

\begin{karte}{Definition nilpotent}
    Ein Endomorphismus \(f\), für den es ein \(n \in \N\) mit
    \[ f^n = \underbrace{f \circ \ldots \circ f}_{n \text{ mal}} = 0 \]
    gibt, heißt nilpotent.\\
    \(\spec f = \set{0}\)
\end{karte}

\begin{karte}{Nilpotenter Endomorphismus lineare Unabhängigkeit}
    Sei \(\abb{f}{V}{V}\) nilpotent. Sei \(v \in V\setminus \set{0}\)
    und \(d \in \N\) so, dass \(f^d(v) = 0\), aber \(f^{d-1}(v) \neq 0\).
    Dann ist \((v,f(v),\ldots, f^{d-1}(v))\) linear unabhängig.
\end{karte}

\begin{karte}{Ähnlichkeit zur Jordan-Normalform Matrizen}
    Sei \(K\) algebraisch abgeschlossen. Dann ist jede Matrix \(A \in M_n(K)\)
    zu einer Matrix in Jordan-Normalform ähnlich, d. h. es gibt \(S \in GL_n(K)\)
    und \(\lambda_1, \ldots, \lambda_k \in K\) und \(n_1, \ldots, n_k \in \N\)
    mit \(n = n_1 + \ldots + n_k\) so, dass
    \[ S^{-1}AS = J_{n_1}(\lambda_1) \boxplus \ldots \boxplus 
    J_{n_k}(\lambda_k) \in M_n(K). \]
    Die Matrix auf der rechten Seite wird dann eine Jordan-Normalform von \(A\)
    genannt. Die Basis \((Se_1, \ldots, Se_n)\) wird Jordan-Basis genannt.
\end{karte}

\begin{karte}{Ähnlichkeit zur Jordan-Normalform Endomorphismen}
    Sei \(K\) algebraisch abgeschlossen. Sei \(\abb{f}{V}{V}\) ein Endomorphismus
    eines endlich-dimensionalen \(K\)-Vektorraums \(V\). Dann gibt es eine Basis
    \(B\) von \(V\) - genannt Jordan-Basis - derart, dass
    \[ M_ {B,B}(f) = J_{n_1}(\lambda_1) \boxplus \ldots \boxplus J_{n_k}(\lambda_k)
    \in M_n(K) \]
\end{karte}

\begin{karte}{Satz Ähnlichkeitsklassifikation}
    Sei \(K\) algebraisch abgeschlossen. Zwei Matrizen in \(M_n(K)\) sind genau dann 
    ähnlich, wenn sie Jordan-Normalformen besitzen, die bis auf eine Permutation der 
    Jordankästchen übereinstimmen.\\
    Sei \(F\) ein Teilkörper von \(K\). Sind \(A, B \in M_n(F)\) über \(K\)
    ähnlich, dann auch über \(F\).
\end{karte}

\subsection*{Eindeutigkeit \& Struktur der JNF}

\begin{karte}{Verallgemeinerte geometrische Vielfachheiten}
    Sei \(\abb{f}{V}{V}\) ein Endomorphismus eines endlich-dimensionalen \\
    \(K\)-Vektorraums. Ist \(\lambda \in \spec f\), so bezeichnen wir die Zahlen
    \[ m_{\lambda, s}(f) = \dim_K \ker(f-\lambda \id_V)^s \text{ für } s \in \N \]
    als verallgemeinerte geometrische Vielfachheiten des Eigenwerts
    \(\lambda\) von \(f\). Analog definiert man \(m_{\lambda, s}(A)\) für eine Matrix
    \(A \in M_n(K)\).\\
    Es gilt \(m_{\lambda, s}(J_n(\lambda)) = \min\set{s,n}\).
\end{karte}

\begin{karte}{Ähnlichkeit und verallgemeinerte geometrische Vielfachheiten}
    Sind \(A, B \in M_n(K)\) ähnlich, so ist
    \[ m_{\lambda, s}(A) = m_{\lambda, s}(B) \; \forall s \in \N_0, \lambda \in K \]
\end{karte}

\begin{karte}{Definition Hauptraum}
    Sei \(f\) ein Endomorphismus und \(\lambda\) ein Eigenwert von \(f\). Dann ist
    \[ H_\lambda(f) := \bigcup_{s \in \N} \ker (f-\lambda \id)^s \]
    der Hauptraum von \(f\) zum Eigenwert \(\lambda\).\\
    Sei \(e\) die algebraische Vielfachheit von \(\lambda\), dann
    \[ H_\lambda(f) = \ker(f-\lambda\id)^e. \]
    \[ \dim H_\lambda(f) = m_{\lambda, e}(f) = e. \]
\end{karte}

\begin{karte}{Hauptraumzerlegung}
    Sei \(K\) ein beliebiger Körper und \(V\) ein endlich-dimensionaler \(K\)-Vektorraum.
    Sei \(\abb{f}{V}{V}\) ein Endomorphismus. Wenn \(\chi_f\) in Linearfaktoren
    zerfällt und \(\lambda_1, \ldots, \lambda_r\) die verschiedenen Eigenwerte
    von \(f\) sind, dann gilt
    \[ V = H_{\lambda_1}(f) \oplus \cdots \oplus H_{\lambda_r}(f) \]
\end{karte}

\begin{karte}{Interpretation allgebraische und geometrische Vielfachheit}
    Sei \(A \in M_n(K)\) und \(\lambda \in \spec A\). Weiter sei eine JNF von \(A\)
    gegeben.
    \begin{enumerate}
        \item Die algebraische Vielfachheit von \(\lambda\) ist gleich der Summe der
        Längen aller \(\lambda\)-Jordankästchen.
        \item Die geometrische Vielfachheit \(m_{\lambda,1}(A)\) von \(\lambda\) ist
        gleich der Anzahl aller \(\lambda\)-Jordankästchen.
        \item Weiter ist
        \[ 2m_{\lambda, s}(A)-m_{\lambda, s-1}(A) - m_{\lambda,s+1}(A) \]
        die Anzahl der \(\lambda\)-Jordankästchen der Länge \(s\).
    \end{enumerate}
\end{karte}

\begin{karte}{Eindeutigkeit JNF}
    Die Jordan-Normalform einer Matrix ist bis auf Permutation der \\
    Jordankästchen eindeutig.
\end{karte}

\subsection*{Existenz der JNF}

\begin{karte}{JNF von nilpotenten Endomorphismen}
    Sei \(\abb{f}{V}{V}\) ein nilpotenter Endomorphismus eines endlich-dimensionalen
    \(K\)-Vektorraums \(V\). Dann gibt es eine Basis \(B\) von \(V\) derart, dass
    \[ M_{B,B}(f) = J_{n_1}(0) \boxplus \cdots \boxplus J_{n_k}(0) \in M_n(K). \]
\end{karte}

\begin{karte}{Nilpotente Endomorphismen Untervektorraumkomplemente}
    Sei \(\abb{f}{V}{V}\) ein nilpotenter Endomorphismus eines endlich-dimensionalen
    \(K\)-Vektorraums \(V\). Sei \(f^d = 0\) und \(u \in V\) ein Vektor mit
    \(f^{d-1}(u) \neq 0\). Dann besitzt der \(f\)-zyklische Untervektorraum
    \(U := \langle \set{u, f(u), \ldots, f^{d-1}(u)}\rangle\) ein \(f\)-invariantes
    Komplement.
\end{karte}

\subsection*{Beispiele und Berechnungen}

\begin{karte}{JNF Rechenverfahren für nilpotente Endomorphismen}
    \begin{enumerate}
        \item Bestimme \(n_1 \in \N\) so, dass \(f^{n_1} = 0 \) und
        \(f^{n_1-1} \neq 0\).
        \item Bestimme eine Basis \((v^{(1)}_1,\ldots,v^{(1)}_{a_1})\)
        des Komplements von \(\ker f^{n_1-1}\).
        \item Definiere \(B_1\) als das Tupel\\
        \( \left(v_1^{(1)}, \ldots, f^{n_1-1}\left(v_1^{(1)}\right), \ldots,
        v_{a_1}^{(1)}, \ldots, f^{n_1-1}\left(v_{a_1}^{(1)}\right)\right) \)\\
        Dann ust \(V_1:=\langle B_1 \rangle \; f\)-invariant und
        \( M_{B_1,B_1}(f\vert_{V_1}) = J_{n_1}(0) \boxplus \cdots \boxplus J_{n_1}(0) \)
        \item Sei \(B_i\) gegeben. Bestimme eine Basis \((v_1^{(i+1)},\ldots, v_{a_{i+1}}^{(i+1)})\)
        des Komplements des Untervektorraums
        \[ \ker(f^{n_{i+1}-1}) + (V_1 \cap \ker(f^{n_{i+1}})) \subset \ker f^{n_{i+1}} \]
        und setze:
        \[ B_{i+1} = \left(v_{i+1}^{(i+1)}, \ldots, f^{n_{i+1}-1}\left(v_{i+1}^{(i+1)}\right), \ldots,
        v_{a_{i+1}}^{(i+1)}, \ldots, f^{n_{i+1}-1}\left(v_{a_{i+1}}^{(i+1)}\right)\right) \]
        \[ V_{i+1} = V_i + \langle B_{i+1} \rangle \]
        \item Die Konkatenation von den \(B_i\) ist eine Jordan-Basis für \(f\).
    \end{enumerate}    
\end{karte}

\begin{karte}{JNF Rechenverfahren allgemein}
    Sei \(f\) ein Endomorphismus eines \(n\)-dimensionalen \(K\)-Vektorraums \(V\),
    für den \(\chi_f(X)\) in Linearfaktoren zerfällt.
    \begin{enumerate}
        \item Man zerlege das charakteristische Polynom von \(f\) in
        Linearfaktoren
        \[ \chi_f(X) = (-1)^n (X-\lambda_1)^{n_1}\cdots (X-\lambda_k)^{n_k} \]
        mit verschiedenen \(\lambda_1, \ldots, \lambda_k \in K\)
        \item Für jedes \(i \in \set{i,\ldots,k}\) berechne gemäß Rechenverfahren
        1.58 eine Jordan-Basis \(B_i\) des nilpotenten Endomorphismus
        \( (f-\lambda \id)\vert_{H_{\lambda_i}(f)} \).
        \item Die Konkatenation der \(B_i\) bildet eine Jordan-Basis von \(f\),
        weil \(V = H_{\lambda_1}(f) \oplus \cdots \oplus H_{\lambda_r}(f)\).
    \end{enumerate}
\end{karte}

\subsection*{Anwendung der Jordan-Normalform}

\begin{karte}{Ähnlichkeit transponierte Matrix}
    Sei \(K\) ein beliebiger Körper und \(A \in M_n(K)\). Dann ist 
    \(A\) zur transponierten Matrix \(A^T\) ähnlich.
\end{karte}

\end{document}