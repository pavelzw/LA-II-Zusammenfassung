\documentclass[main.tex]{subfiles}

\begin{document}

\section*{Geometrie von \(\R, \C\)-Vektorräumen}
\subsection*{Euklidische, unitäre VR}

\begin{karte}{Definition Biliearform}
     Sei \(K\) ein Körper und \(V\) ein \(K\)-Vektorraum. Eine
     Biliearform auf \(V\) ist eine bilieare Abbildung 
     \(V \times V \rightarrow K\).\\
     Eine Biliearform \(\abb{b}{V\times V}{K}\) heißt symmetrisch,
     wenn
     \[ b(x,y) = b(y,x) \]
     für alle \(x, y \in V \) gilt.
\end{karte}

\begin{karte}{Definitheit von Biliearformen}
     Eine Biliearform \(b\) auf einem reellen Vektorraum \(V\) heißt
     \begin{description}
         \item[positiv definit,] falls \(b(x,x) > 0 \; \forall x \in V \setminus \set{0}\)
         \item[positiv semidefinit,] falls \(b(x,x) \geq 0 \; \forall x \in V\)
         \item[negativ definit,] falls \(b(x,x) < 0 \; \forall x \in V \setminus \set{0}\)
         \item[negativ semidefinit,] falls \(b(x,x) \leq 0 \; \forall x \in V\)
         \item[indefinit,] falls es \(x, y \in V \setminus \set{0}\) mit
         \(b(x,x) > 0\) und \(b(y,y) < 0\) gibt.  
     \end{description}
\end{karte}

\begin{karte}{Definition Skalarprodukt}
    Sei \(V\) ein \(\R\)-Vektorraum. Ein Skalarprodukt auf \(V\) ist eine
    symmetrische, positiv definite Biliearform.\\
    Eine positive definite hermitesche Sesquilinearform heißt hermitesches Skalarprodukt.
\end{karte}

\begin{karte}{Definition Sesquilinearform}
    Sei \(V\) ein \(\C\)-Vektorraum. Eine Abbildung \(\abb{b}{V\times V}{\C}\)
    heißt Sesquilinearform, wenn für jedes \(x\in V\) die Abbildung
    \[ \abb{b(x, \cdot)}{V}{\C} \]
    \(\C\)-Linear ist und die Abbildung
    \[ \abb{b(\cdot, x)}{V}{\C} \]
    \(\R\)-Linear ist und für jedes \(\lambda \in \C\) und \(y \in V\) gilt
    \[ b(\lambda y, x) = \overline{\lambda} b(y,x) \]
\end{karte}

\begin{karte}{Definition konjugiert linear}
    Eine rell lineare Abbildung \(\abb{f}{V}{W}\) zwischen komplexen Vektorräumen,
    die \(f(\lambda v) = \overline{\lambda}f(v)\) für jedes \(\lambda \in \C\)
    und \(v \in V\) erfüllt, heißt auch konjugiert linear.
\end{karte}

\begin{karte}{Definition hermitesche Sesquilinearform}
    Sei \(\abb{b}{V\times V}{\C}\) eine Sesquilinearform. Dann heißt \(b\) hermitesch,
    wenn für alle \(x,y \in V\)
    \[ b(x,y) = \overline{b(y,x)} \]
    gilt.\\
    Eine hermitesche Sesquilinearform \(b\) auf \(V\) heißt positiv definit,
    wenn \(b(x,x) > 0\) für alle \(x \in V \setminus \set{0}\)
\end{karte}

\begin{karte}{Definition adjungierte Matrix}
    Für eine Matrix \(A \in M_{m,n}(\C)\) ist ihre adjungierte Matrix \(A^H \in M_{n,m}(\C)\)\\
    diejenige Matrix, die man erhält, wenn man \(A\) transponiert und alle Matrixeinträge 
    komplex konjugiert. Man bezeichnet \(A^H\) auch als das hermitesch Transponierte von \(A\).\\
    Es gelten die Rechenregeln:\\
    \[ \overline{A} + \overline{B} = \overline{A + B} \]
    \[ \overline{A} \cdot \overline{B} = \overline{AB} \]
    \[ A^H + B^H = (A + B)^H \]
    \[ B^H \cdot A^H = (A \cdot B)^H \]
\end{karte}

\begin{karte}{Definition euklidischer und unitärer Vektorraum}
    Ein euklidischer Vektorraum ist ein Paar bestehend aus einem reellen Vektorraum \(V\)
    und einem Skalarprodukt auf \(V\).\\
    Ein unitärer Vektorraum ist ein Paar bestehend aus einem komplexen Vektorraum \(V\) und
    einem hermiteschen Skalarprodukt auf \(V\).
\end{karte}

\begin{karte}{Definition Norm}
    Sei \(K\) der Körper der reellen oder komplexen Zahlen. Sei \(V\) ein \(K\)-Vektorraum.
    Eine Norm auf \(V\) ist eine Abbildung
    \[ \abb{\norm{\cdot}}{V}{[0, \infty)} \]
    mit folgenden Eigenschaften:
    \begin{description}
        \item[Homogenität] Für alle \(\lambda \in K\) und \(v \in V\) ist 
        \[ \norm{\lambda v} = \abs{\lambda} \norm{v}. \]
        \item[Definitheit] Es ist \(\norm{v} = 0\) genau dann, wenn \(v = 0\).
        \item[Dreiecksungleichung] Für alle \(v, w \in V\) ist
        \[ \norm{v+w} \leq \norm{v} + \norm{w}. \]   
    \end{description}
\end{karte}

\begin{karte}{Normierter Vektorraum}
    Sei \(K \in \set{\R, \C}\). Ein normierter \(K\)-Vektorraum ist ein Paar bestehend
    aus einem \(K\)-Vektorraum \(V\) und einer Norm auf \(V\).\\
    Normierte Vektorräume sind gleichzeitig metrische Räume mit
    \[ \abb{d}{V\times V}{[0, \infty)}, (x, y) \mapsto \norm{x-y}. \]
\end{karte}

\begin{karte}{Definition induzierte Norm}
    Sei \((V, \scalarprod{\cdot}{\cdot})\) ein euklidischer oder unitärer Vektorraum.
    Dann definieren wir durch
    \[ \abb{\norm{\cdot}}{V}{\R}, x \mapsto \sqrt{\scalarprod{x}{x}} \]
    die von \(\scalarprod{\cdot}{\cdot}\) induzierte Norm auf \(V\).\\
    \(\norm{\cdot}_2\) ist die von \(\scalarprod{\cdot}{\cdot}_2\) induzierte Norm und
    wird als euklidische Norm bezeichnet.\\
    Die Supremumsnorm ist definiert durch 
    \[ \abb{\norm{\cdot}_\infty}{\R^n}{[0,\infty)}, x \mapsto \max\set{\abs{x_1}, \ldots, \abs{x_n}} \]
\end{karte}

\begin{karte}{Ungleichung von Cauchy-Schwarz}
    Sei \((V, \scalarprod{\cdot}{\cdot})\) ein euklidischer oder unitärer Vektorraum
    und \(\norm{\cdot}\) die Norm. Dann gilt
    \[ \abs{\scalarprod{x}{y}} \leq \norm{x} \cdot \norm{y}. \]
    Gleichheit tritt genau dann ein, wenn \(x\) und \(y\) linear abhängig sind.
\end{karte}

\begin{karte}{Polarisierung}
    Sei \((V, \scalarprod{\cdot}{\cdot})\) ein euklidischer oder unitärer Vektorraum mit induzierter
    Norm \(\norm{\cdot}\).
    \begin{enumerate}
        \item Im euklidischen Fall gilt für alle \(x, y \in V\)
        \[ \scalarprod{x}{y} = \frac{1}{2}(\norm{x+y}^2-\norm{x}^2-\norm{y}^2). \]
        \item Im unitären Fall gilt für alle \(x, y \in V\)
        \[ \scalarprod{x}{y} = \frac{1}{4}(\norm{x+y}^2-\norm{x-y}^2) 
        - \frac{i}{4}(\norm{x+iy}^2-\norm{x-iy}^2). \]
    \end{enumerate}
\end{karte}

\begin{karte}{Definition Winkel}
    Sei \((V, \scalarprod{\cdot}{\cdot})\) ein euklidischer Vektorraum mit induzierter Norm \(\norm{\cdot}\).
    Seien \(x, y \in V \setminus \set{0}\). Dann ist der Winkel zwischen \(x\) und \(y\) definiert als
    \[ \angle (x,y) := \arccos \frac{\scalarprod{x}{y}}{\norm{x} \norm{y}} \in [0, \pi]. \]
\end{karte}

\begin{karte}{Definition lineare isometrische Einbettung und lineare Isometrie}
    Seien \((V, \norm{\cdot})\) und \((V', \norm{\cdot}')\) normierte Vektorräume über 
    \(K \in \set{\R, \C}\).\\
    Eine lineare isometrische Einbettung \((V, \norm{\cdot}) \rightarrow (V', \norm{\cdot}')\)
    ist eine \(K\)-lineare Abbildung \(\abb{f}{V}{V'}\) mit
    \[ \norm{f(x)}' = \norm{x} \; \forall x \in V. \]
    Eine lineare Isometrie \((V, \norm{\cdot}) \rightarrow (V', \norm{\cdot}')\) ist eine
    bijektive lineare isometrische Einbettung \((V, \norm{\cdot}) \rightarrow (V', \norm{\cdot}')\).\\
    Lineare Isometrien sind winkeltreu, d. h. für alle \(x, y \in V \setminus \set{0}\) gilt
    \[ \angle (f(x), f(y)) = \angle (x,y) \]
\end{karte}

\begin{karte}{Definition orthogonale Matrix}
    Eine Matrix \(A \in \GL_n(\R)\) heißt orthogonal, falls
    \[ A^{-1} = A^T. \]
    Eine Matrix \(A \in \GL_n(\C)\) heißt unitär, alls
    \[A^{-1} = A^H.\]
\end{karte}

\begin{karte}{Wichtige Untergruppen von \(\GL_n\)}
    Die folgenden Teilmengen von \(\GL_n(\R)\) bzw. \(\GL_n(\C)\) sind Untergruppen:
    \begin{align*}
        O(n) &:= \set{A \in \GL_n(\R) \vert A \text{ orthogonal}} &\text{orthogonale Gruppe}\\
        U(n) &:= \set{A \in \GL_n(\R) \vert A \text{ unitär}} &\text{unitärer Gruppe}\\
        SO(n) &:= \set{A \in O(n) \vert \det A = 1} &\text{spezielle orthogonale Gruppe}\\
        SU(n) &:= \set{A \in U(n) \vert \det A = 1} &\text{spezielle unitäre Gruppe}
    \end{align*}
\end{karte}

\begin{karte}{Eigenschaften lineare Isometrie}
    Sei \(B = (v_1, \ldots, v_n)\) eine Orthonormalbasis des euklidischen oder unitären 
    Vektorraums \(V\). Sei \(\abb{f}{V}{V}\) ein Endomorphismus.
    \begin{enumerate}
        \item Es ist \(f\) eine lineare Isometrie genau dann, wenn \((f(v_1), \ldots, f(v_n))\)
        eine Orthonormalbasis ist.
        \item Es ist \(f\) eine lineare Isometrie genau dann, wenn \(M_{B,B}(f)\) orthogonal 
        bzw. unitär ist.
    \end{enumerate}
\end{karte}

\end{document}