\documentclass[main.tex]{subfiles}

\begin{document}

\section*{Geometrie von \(\R, \C\)-Vektorräumen}
\subsection*{Euklidische, unitäre VR}

\begin{karte}{Definition Bilinearform}
    Sei \(K\) ein Körper und \(V\) ein \(K\)-Vektorraum. 
    Eine Bilinearform auf \(V\) ist eine bilineare Abbildung 
    \(V \times V \rightarrow K\).\\
\end{karte}

\begin{karte}{Symmetrische Bilinearform}
    Eine Bilinearform \(\abb{b}{V\times V}{K}\) heißt symmetrisch,
    wenn
    \[ b(x,y) = b(y,x) \]
    für alle \(x, y \in V \) gilt. Eine Matrix heißt symmetrisch, 
    falls \(A^T = A\).\\
    \( b \) ist genau dann symmetrisch, wenn \( M_C(b) \) 
    symmetrisch ist.
\end{karte}

\begin{karte}{Definitheit von Bilinearformen}
    Eine Bilinearform \(b\) auf einem reellen Vektorraum \(V\) heißt
    \begin{description}
        \item[positiv definit,] falls \(b(x,x) > 0 \; \forall x \in V \setminus \set{0}\)
        \item[positiv semidefinit,] falls \(b(x,x) \geq 0 \; \forall x \in V\)
        \item[negativ definit,] falls \(b(x,x) < 0 \; \forall x \in V \setminus \set{0}\)
        \item[negativ semidefinit,] falls \(b(x,x) \leq 0 \; \forall x \in V\)
        \item[indefinit,] falls es \(x, y \in V \setminus \set{0}\) mit
        \(b(x,x) > 0\) und \(b(y,y) < 0\) gibt.  
    \end{description}
    Eine Matrix \(A\) heißt positiv definit, wenn
    die Bilinearform \( b(A) \) auf \(\R^n\) 
    positiv definit ist.
\end{karte}

\begin{karte}{Definition Skalarprodukt}
    Sei \(V\) ein \(\R\)-Vektorraum. Ein Skalarprodukt auf \(V\) ist eine
    symmetrische, positiv definite Bilinearform.\\
    Eine positiv definite hermitesche Sesquilinearform heißt hermitesches Skalarprodukt.
\end{karte}

\begin{karte}{Definition Sesquilinearform}
    Sei \(V\) ein \(\C\)-Vektorraum. Eine Abbildung \(\abb{b}{V\times V}{\C}\)
    heißt \textit{Sesquilinearform}, wenn für jedes \(x\in V\) die Abbildung
    \[ \abb{b(x, \cdot)}{V}{\C} \]
    \(\C\)-linear ist und die Abbildung
    \[ \abb{b(\cdot, x)}{V}{\C} \]
    \(\R\)-linear ist und für jedes \(\lambda \in \C\) und \(y \in V\) gilt
    \[ b(\lambda y, x) = \overline{\lambda} b(y,x). \]
\end{karte}

\begin{karte}{Definition konjugiert linear}
    Eine reelle lineare Abbildung \(\abb{f}{V}{W}\) zwischen komplexen Vektorräumen,
    die \(f(\lambda v) = \overline{\lambda}f(v)\) für jedes \(\lambda \in \C\)
    und \(v \in V\) erfüllt, heißt auch \textit{konjugiert linear}.
\end{karte}

\begin{karte}{Definition hermitesche Sesquilinearform}
    Sei \(\abb{b}{V\times V}{\C}\) eine Sesquilinearform. 
    Dann heißt \(b\) \textit{hermitesch},
    wenn für alle \(x,y \in V\)
    \[ b(x,y) = \overline{b(y,x)} \]
    gilt.\\
    Eine hermitesche Sesquilinearform \(b\) auf \(V\) heißt positiv definit,
    wenn \(b(x,x) > 0\) für alle \(x \in V \setminus \set{0}\).
\end{karte}

\begin{karte}{Definition adjungierte Matrix}
    Für eine Matrix \(A \in M_{m,n}(\C)\) ist ihre 
    adjungierte Matrix \(A^H \in M_{n,m}(\C)\)
    diejenige Matrix, die man erhält, wenn man 
    \(A\) transponiert und alle Matrixeinträge 
    komplex konjugiert. Man bezeichnet \(A^H\) 
    auch als das hermitesch Transponierte von \(A\).\\
    Es gelten die Rechenregeln:\\
    \[ \overline{A} + \overline{B} = \overline{A + B} \]
    \[ \overline{A} \cdot \overline{B} = \overline{AB} \]
    \[ A^H + B^H = (A + B)^H \]
    \[ B^H \cdot A^H = (A \cdot B)^H \]
\end{karte}

\begin{karte}{Definition euklidischer und unitärer Vektorraum}
    Ein euklidischer Vektorraum ist ein Paar bestehend aus einem reellen Vektorraum \(V\)
    und einem Skalarprodukt auf \(V\).\\
    Ein unitärer Vektorraum ist ein Paar bestehend aus einem komplexen Vektorraum \(V\) und
    einem hermiteschen Skalarprodukt auf \(V\).
\end{karte}

\begin{karte}{Definition Norm}
    Sei \(K\) der Körper der reellen oder komplexen Zahlen. Sei \(V\) ein \(K\)-Vektorraum.
    Eine Norm auf \(V\) ist eine Abbildung
    \[ \abb{\norm{\cdot}}{V}{[0, \infty)} \]
    mit folgenden Eigenschaften:
    \begin{description}
        \item[Homogenität] Für alle \(\lambda \in K\) und \(v \in V\) ist 
        \[ \norm{\lambda v} = \abs{\lambda} \norm{v}. \]
        \item[Definitheit] Es ist \(\norm{v} = 0\) genau dann, wenn \(v = 0\).
        \item[Dreiecksungleichung] Für alle \(v, w \in V\) ist
        \[ \norm{v+w} \leq \norm{v} + \norm{w}. \]   
    \end{description}
\end{karte}

\begin{karte}{Normierter Vektorraum}
    Sei \(K \in \set{\R, \C}\). Ein normierter \(K\)-Vektorraum ist ein Paar bestehend
    aus einem \(K\)-Vektorraum \(V\) und einer Norm auf \(V\).\\
    Normierte Vektorräume sind gleichzeitig metrische Räume mit
    \[ \abb{d}{V\times V}{[0, \infty)}, (x, y) \mapsto \norm{x-y}. \]
\end{karte}

\begin{karte}{Definition induzierte Norm}
    Sei \((V, \scalarprod{\cdot}{\cdot})\) ein euklidischer oder unitärer Vektorraum.
    Dann definieren wir durch
    \[ \abb{\norm{\cdot}}{V}{\R}, \quad x \mapsto \sqrt{\scalarprod{x}{x}} \]
    die von \(\scalarprod{\cdot}{\cdot}\) induzierte Norm auf \(V\).\\
    \(\norm{\cdot}_2\) ist die von \(\scalarprod{\cdot}{\cdot}_2\) induzierte Norm und
    wird als euklidische Norm bezeichnet.\\
    Die Supremumsnorm ist definiert durch 
    \[ \abb{\norm{\cdot}_\infty}{\R^n}{[0,\infty)}, \quad
    x \mapsto \max\set{\abs{x_1}, \ldots, \abs{x_n}}. \]
\end{karte}

\begin{karte}{Ungleichung von Cauchy-Schwarz}
    Sei \((V, \scalarprod{\cdot}{\cdot})\) ein euklidischer oder unitärer Vektorraum
    und \(\norm{\cdot}\) eine Norm. Dann gilt
    \[ \abs{\scalarprod{x}{y}} \leq \norm{x} \cdot \norm{y}. \]
    Gleichheit tritt genau dann ein, wenn \(x\) und \(y\) linear abhängig sind.
\end{karte}

\begin{karte}{Polarisierung}
    Sei \((V, \scalarprod{\cdot}{\cdot})\) ein euklidischer oder unitärer Vektorraum mit induzierter
    Norm \(\norm{\cdot}\).
    \begin{enumerate}
        \item Im euklidischen Fall gilt für alle \(x, y \in V\)
        \[ \scalarprod{x}{y} = \frac{1}{2}(\norm{x+y}^2-\norm{x}^2-\norm{y}^2). \]
        \item Im unitären Fall gilt für alle \(x, y \in V\)
        \[ \scalarprod{x}{y} = \frac{1}{4}(\norm{x+y}^2-\norm{x-y}^2) 
        - \frac{i}{4}(\norm{x+iy}^2-\norm{x-iy}^2). \]
    \end{enumerate}
\end{karte}

\begin{karte}{Definition Winkel}
    Sei \((V, \scalarprod{\cdot}{\cdot})\) ein euklidischer Vektorraum mit induzierter Norm \(\norm{\cdot}\).
    Seien \(x, y \in V \setminus \set{0}\). Dann ist der Winkel zwischen \(x\) und \(y\) definiert als
    \[ \angle (x,y) := \arccos \frac{\scalarprod{x}{y}}{\norm{x} \norm{y}} \in [0, \pi]. \]
\end{karte}

\begin{karte}{Definition lineare isometrische Einbettung und lineare Isometrie}
    Seien \((V, \norm{\cdot})\) und \((V', \norm{\cdot}')\) normierte Vektorräume über 
    \(K \in \set{\R, \C}\).\\
    Eine lineare isometrische Einbettung \((V, \norm{\cdot}) \rightarrow (V', \norm{\cdot}')\)
    ist eine \(K\)-lineare Abbildung \(\abb{f}{V}{V'}\) mit
    \[ \norm{f(x)}' = \norm{x} \; \forall x \in V. \]
    Eine lineare Isometrie \((V, \norm{\cdot}) \rightarrow (V', \norm{\cdot}')\) ist eine
    bijektive lineare isometrische Einbettung \((V, \norm{\cdot}) \rightarrow (V', \norm{\cdot}')\).\\
    Lineare Isometrien sind winkeltreu, d. h. für alle \(x, y \in V \setminus \set{0}\) gilt
    \[ \angle (f(x), f(y)) = \angle (x,y). \]
\end{karte}

\begin{karte}{Definition orthogonale Matrix}
    Eine Matrix \(A \in \GL_n(\R)\) heißt orthogonal, falls
    \[ A^{-1} = A^T. \]
    Eine Matrix \(A \in \GL_n(\C)\) heißt unitär, falls
    \[A^{-1} = A^H.\]
\end{karte}

\begin{karte}{Wichtige Untergruppen von \(\GL_n\)}
    Die folgenden Teilmengen von \(\GL_n(\R)\) bzw. \(\GL_n(\C)\) sind Untergruppen:
    \begin{align*}
        \mathrm{O}(n) &:= \set{A \in \GL_n(\R) \;\vert\; A \text{ orthogonal}} &\text{orthogonale Gruppe}\\
        \mathrm{U}(n) &:= \set{A \in \GL_n(\R) \;\vert\; A \text{ unitär}} &\text{unitäre Gruppe}\\
        \mathrm{SO}(n) &:= \set{A \in \mathrm{O}(n) \;\vert\; \det A = 1} &\text{spezielle orthogonale Gruppe}\\
        \mathrm{SU}(n) &:= \set{A \in \mathrm{U}(n) \;\vert\; \det A = 1} &\text{spezielle unitäre Gruppe}
    \end{align*}
\end{karte}

\begin{karte}{Eigenschaften lineare Isometrie}
    Sei \(B = (v_1, \ldots, v_n)\) eine Orthonormalbasis des euklidischen oder unitären 
    Vektorraums \(V\). Sei \(\abb{f}{V}{V}\) ein Endomorphismus.
    \begin{enumerate}
        \item Es ist \(f\) eine lineare Isometrie genau dann, wenn \((f(v_1), \ldots, f(v_n))\)
        eine Orthonormalbasis ist.
        \item Es ist \(f\) eine lineare Isometrie genau dann, wenn \(M_{B,B}(f)\) orthogonal 
        bzw. unitär ist.
    \end{enumerate}
\end{karte}

\subsection*{Matrizenkalkül von Bilinearformen}

\begin{karte}{Bilinearformen Vektorraum}
    \( \Bil_K(V) \) ist die Menge der 
    Bilinearformen auf \( V \). \\
    Sie bildet einen \( K \)-Vektorraum.\\
    Sei \( V \) ein \(n\)-dimensionaler \(K\)-Vektorraum 
    und \( C=(c_1, \ldots, c_n) \) eine Basis von \(V\).
    Dann sind  
    \[ \abb{\phi}{\Bil_K(V)}{M_n(K)},\quad b\mapsto M_C(b) := (b(c_j, c_k))_{j,k\in\set{1,\ldots,n}} \]
    \[ \abb{\psi}{M_n(K)}{\Bil_K(V)},\quad A \mapsto b_C(A) \]
    zueinander inverse Isomorphismen.
\end{karte}

\begin{karte}{Basiswechsel Bilinearformen}
    Sei \(V\) ein \(n\)-dimensionaler \(K\)-Vektorraum. 
    Sei \( b\in \Bil_k(V) \). Sei \(C\) eine Basis 
    von \(V\) und \(A\in M_n(K)\). Dann sind 
    äquivalent:
    \begin{enumerate}
        \item Es gibt eine Matrix \(S\in \GL_n(K)\) mit 
        \[ A=S^T \cdot M_C(b) \cdot S. \]
        \item Es gibt eine Basis \( D \) von \(V\) mit
        \(A=M_D(b)\).
    \end{enumerate}
\end{karte}

\begin{karte}{Sesquilinearformen Vektorraum}
    \( \Sil_\C(V) \) ist die Menge der 
    Sesquilinearformen auf \( V \). \\
    Sie bildet einen \( \C \)-Vektorraum.\\
    Sei \( V \) ein \(n\)-dimensionaler \(\C\)-Vektorraum 
    und \( C=(c_1, \ldots, c_n) \) eine Basis von \(V\).
    Dann sind  
    \[ \abb{\phi}{\Sil_\C(V)}{M_n(\C)}, \quad
    b\mapsto M_C(b) := (b(c_j, c_k))_{j,k\in\set{1,\ldots,n}} \]
    \[ \abb{\psi}{M_n(\C)}{\Sil_\C(V)},\quad A \mapsto b^\C_C(A) \]
    zueinander inverse Isomorphismen.
\end{karte}

\begin{karte}{Basiswechsel Sesquilinearform}
    Sei \(V\) ein \(n\)-dimensionaler \(\C\)-Vektorraum. 
    Sei \(b \in \Sil_\C(V)\). Sei \(C\) eine Basis von 
    \(V\) und \(A\in M_n(\C)\). Dann sind äquivalent:
    \begin{enumerate}
        \item Es gibt eine Matrix \( S \in \GL_n(\C) \) mit 
        \[ A=S^H \cdot M_C(b) \cdot S. \]
        \item Es gibt eine Basis \(D\) von \( V \) 
        mit \( A=M_D(b) \).
    \end{enumerate}
\end{karte}

\begin{karte}{Hermitesch}
    Eine Matrix \(A\in M_n(\C)\) ist \textit{hermitesch} 
    oder \textit{selbstadjungiert}, wenn \(A^H=A\).\\
    Eine Sesquilinearform \( b \) ist genau dann 
    hermitesch, wenn \(M_C(b)\) hermitesch ist.
\end{karte}

\subsection*{Orthogonalität}

\begin{karte}{Orthogonal}
    Sei \( (V, \scalarprod{\cdot}{\cdot}) \) ein 
    euklidischer oder unitärer Vektorraum. 
    \begin{itemize}
        \item Zwei Vektoren \(v,w\in V\) heißen 
        \textit{orthogonal}, falls 
        \( \scalarprod{v}{w} = 0 \). In diesem Fall 
        schreiben wir auch \(v\bot w\).
        \item \( A, B \subset V \) heißen \textit{orthogonal}, 
        wenn für \( v\in A,\ w\in B \) stets \(v\bot w\) gilt.
        Wir schreiben \(A\bot B\).
        \item Ist \( A\subset V \), so nennen wir 
        \[ A^\bot := \set{ v\in V \;\vert\; v\bot w \text{ für alle } w\in A } \]
        das \textit{orthogonale Komplement} von \(A\). Das orthogonale 
        Komplement ist ein Untervektorraum. \( U^\bot \) ist 
        komplementär zu \(U\).
    \end{itemize}
\end{karte}

\begin{karte}{Orthonormalsystem, Orthonormalbasis}
    Sei \( (V, \scalarprod{\cdot}{\cdot}) \) ein 
    euklidischer oder unitärer Vektorraum. 
    \begin{enumerate}
        \item Eine Familie \( (v_i)_{i\in I} \) 
        von Vektoren in \(V\) ist ein 
        \textit{Orthonormalsystem}, wenn für alle 
        \( i,j \in I \) gilt.
        \[ \scalarprod{v_i}{v_j} = \delta_{i,j}. \]
        \item Eine Orthonormalbasis von 
        \( (V, \scalarprod{\cdot}{\cdot}) \) ist 
        eine Basis von \(V\), die ein Orthonormalsystem ist.
    \end{enumerate}
    Jedes Orthonormalsystem ist linear unabhängig.\\
    Jeder endlich-dimensionale euklidische oder 
    unitäre Vektorraum besitzt eine Orthonormalbasis.
\end{karte}

\begin{karte}{Orthonormalisierungsverfahren von Gram-Schmidt}
   Sei \( (V, \scalarprod{\cdot}{\cdot}) \) ein 
   euklidischer oder unitärer Vektorraum. Es sei 
   \( (v_1,\ldots, v_n) \) eine linear unabhängiges 
   \(n\)-Tupel in \(V\). Dann gibt es ein 
   Orthonormalsystem \( (w_1, \ldots, w_n) \) in \(V\) 
   mit 
   \[ \langle \set{w_1, \ldots, w_k} \rangle 
   = \langle \set{ v_1, \ldots, v_k } \rangle \quad 
   \forall k \in \set{1,\ldots, n}. \]
   Weiter gilt: Das \(n\)-Tupel 
   \( (w_1, \ldots, w_n) \), das rekursiv durch 
   \[ w_1 := \frac{1}{\norm{\tilde{w}_1}} \cdot \tilde{w}_1 
   \text{ und } \tilde{w}_1 := v_1 \]
   und 
   \[ w_{k+1} := \frac{1}{\norm{ \tilde{w}_{k+1} }} 
   \cdot \tilde{w}_{k+1} \text{ und } 
   \tilde{w}_{k+1} := 
   v_{k+1} - \sum_{j=1}^k \scalarprod{w_j}{v_{k+1}} \cdot w_j \]
   für \( k\in \set{1, \ldots, n-1} \) definiert ist, hat 
   diese Eigenschaft.
\end{karte}

\begin{karte}{Isometrie zwischen Vektorräumen}
    Sei \( (V, \scalarprod{\cdot}{\cdot}) \) ein 
    endlich-dimensionaler euklidischer oder 
    unitärer Vektorraum mit Grundkörper 
    \( K\in \set{\R, \C} \). Dann gibt es eine lineare 
    Isometrie zwischen \( (K^n, \scalarprod{\cdot}{\cdot}_2) \)
    und \( (V, \scalarprod{\cdot}{\cdot}) \).
\end{karte}

\begin{karte}{Orthogonale Projektion}
    Sei \( (V,\scalarprod{\cdot}{\cdot}) \) ein 
    euklidischer oder unitärer Vektorraum und 
    sei \( U \subset V \) ein endlich-dimensionaler 
    Untervektorraum. Dann gibt es genau eine lineare 
    Abbildung \( \abb{p_U}{V}{V} \) mit folgenden 
    Eigenschaften:
    \begin{enumerate}
        \item Es gilt \(\Bild(p_U) \subset U\).
        \item Es gilt \( \Bild(p_U - \id_V)\subset U^\bot \).
    \end{enumerate}
    Weiter ist \( p_U(x) = x \) für \( x\in U \). Man 
    bezeichnet \( p_U \) als \textit{orthogonale Projektion} 
    von \(V\) auf \(U\).
\end{karte}

\subsection*{Selbstadjungierte Endomorphismen}

\begin{karte}{Selbstadjungiert}
    Ein Endomorphismus \( \abb{f}{V}{V} \) heißt 
    \textit{selbstadjungiert}, wenn für alle 
    \( x,y\in V \) gilt:
    \[ \scalarprod{x}{f(y)} = \scalarprod{f(x)}{y}. \]
    Eine komplexe Matrix ist selbstadjungiert, wenn 
    \( A^H = A \). Wenn alle Matrixeinträge reell 
    sind, ist sie selbstadjungiert, wenn \(A^T = A\).
    \( M_{B,B}(f) \) ist genau dann selbstadjungiert, 
    wenn \(f\) selbstadjungiert ist.
\end{karte}

\begin{karte}{Adjungierte Abbildung}
    Sei \( \abb{f}{V}{V} \) ein Endomorphismus 
    eines endlich-dimensionalen euklidischen oder unitären 
    Vektorraums. Dann gibt es genau einen Endomorphismus 
    \( \abb{f^H}{V}{V} \) so, dass 
    \[ \scalarprod{x}{f(y)} = \scalarprod{f^H(x)}{y} \] 
    für alle \( x,y\in V \) gilt. \\
    Man bezeichnet \(f^H\) als adjungierte Abbildung 
    von \(f\).\\
    Ein Endomorphismus \(f\) ist genau dann selbstadjungiert, 
    wenn \(f^H = f\).
    Es gilt außerdem 
    \[ M_{B,B}(f^H) = M_{B, B}(f)^H. \]
\end{karte}

\begin{karte}{Rechenregeln adjungierter Abbildungen}
    Seien \(\abb{f,g}{V}{V}\) Endomorphismen eines 
    endlich-dimensionalen euklidischen oder unitären 
    Vektorraums. Seien \( \lambda, \mu\in \C \). Dann 
    gilt 
    \begin{enumerate}
        \item \( (\lambda \cdot f + \mu \cdot g)^H 
        = \overline{\lambda} \cdot f^H + \overline{\mu} \cdot g^H \).
        \item \( (f^H)^H =f \).
    \end{enumerate}
\end{karte}

\begin{karte}{Spektralsatz für selbstadjungierte Endomorphismen}
    Sei \( \abb{f}{V}{V} \) ein selbstadjungierter Endomorphismus 
    eines \\
    endlich-dimensionalen euklidischen oder unitären 
    Vektorraums. Es bezeichne \( K\in \set{\R, \C} \) 
    den Grundkörper von \(V\). Dann gilt 
    \begin{enumerate}
        \item \( \spec_K(f) \subset \R \).
        \item Es gibt eine Orthonormalbasis von \(V\) 
        aus Eigenvektoren von \(f\). Insbesondere ist 
        \(f\) diagonalisierbar.
    \end{enumerate}
\end{karte}

\begin{karte}{Diagonalisierung symmetrischer Matrizen}
    Ist die Matrix \( A\in M_n(\R) \) symmetrisch, 
    so gibt es ein \( S\in \mathrm{O}(n) \) so, dass 
    \[ S^T \cdot A \cdot S = S^{-1} \cdot A\cdot S 
    = \begin{pmatrix}
        \lambda_1 && 0 \\
        & \ddots & \\
        0 && \lambda_n
    \end{pmatrix} \]
    eine Diagonalmatrix mit reellen Einträgen ist.
\end{karte}

\begin{karte}{Diagonalisierung hermitescher Matrizen}
    Ist die Matrix \( A\in M_n(\C) \) hermitesch, so gibt es 
    ein \( S\in \mathrm{U}(n) \) hermitesch so, dass 
    \[ S^H \cdot A \cdot S = S^{-1} \cdot A \cdot S 
    = \begin{pmatrix}
        \lambda_1 && 0 \\
        & \ddots & \\
        0 && \lambda_n
    \end{pmatrix} \]
    eine Diagonalmatrix mit reellen Einträgen ist.
\end{karte}

\begin{karte}{Normale Endomorphismen}
    Wir nennen einen Endomorphismus 
    \( f \) eines euklidischen oder unitären 
    Vektorraums endlicher Dimension \textit{normal}, 
    wenn 
    \[ f^H \circ f = f \circ f^H. \]
    Ist \( f \) normal, dann auch \( \mu \cdot f \) 
    und \( f + \mu \cdot \id_V \). 
\end{karte}

\begin{karte}{Spektralsatz für normale Endomorphismen}
    Sei \( f \) ein Endomorphismus eines endlich-dimensionalen 
    euklidischen oder \\
    unitären Vektorraums, dessen 
    charakterisches Polynom in Linearfaktoren \\
    zerfällt. 
    Dann gibt es genau dann eine Orthonormalbasis aus 
    Eigenvektoren von \(f\), wenn \(f\) normal ist.
\end{karte}

\begin{karte}{Eigenschaften normaler Endomorphismen}
    Sei \(\abb{f}{V}{V}\) normal. Dann gilt: 
    \begin{enumerate}
        \item \( \ker f = \ker f^H \).
        \item Ein Vektor \(v\in V\) ist genau dann ein 
        Eigenvektor von \(f\) zum Eigenwert \(\lambda\), 
        wenn \(v\) ein Eigenvektor von \( f^H \) zum 
        Eigenwert \( \overline{\lambda} \) ist.
    \end{enumerate}
\end{karte}

\begin{karte}{Definitheit symmetrischer Matrizen}
    Sei \( A\in M_n(\R) \) symmetrisch. Dann ist \(A\)
    \begin{itemize}
        \item genau dann positiv definit, 
        wenn \( \spec_\R(A) \subset (0,\infty) \).
        \item genau dann positiv semidefinit, 
        wenn \( \spec_\R(A) \subset [0,\infty) \).
        \item genau dann negativ definit, 
        wenn \( \spec_\R(A) \subset (-\infty, 0) \).
        \item genau dann negativ semidefinit, 
        wenn \( \spec_\R(A) \subset (-\infty, 0] \).
        \item genau dann indefinit, wenn 
        \( \spec_\R(A) \) negative und positive Zahlen 
        enthält.
    \end{itemize}
\end{karte}

\begin{karte}{\(k\)-ter Hauptminor}
    Sei \(A = (a_{i,j}) \in M_n(K)\) eine 
    quadratische Matrix über einem beliebigen Körper. 
    Sei \( 1 \leq k\leq n \). Die Determinante der 
    linken oberen \( k\times k \)-Teilmatrix 
    \( (a_{i,j})_{1\leq i,j \leq k} \) heißt 
    \textit{\(k\)-ter Hauptminor} von \(A\).
\end{karte}

\begin{karte}{Hurwitz-Kriterium}
    Eine reelle symmetrische Matrix \(A\) ist genau dann 
    positiv definit, wenn alle ihre Hauptminoren 
    positiv sind.\\
    \(A\) ist genau dann negativ definit, wenn 
    \(-A\) positiv definit ist oder 
    wenn die Hauptminoren alternierend positiv und 
    negativ sind (startend mit negativ).
\end{karte}

\subsection*{Isometrien}

\begin{karte}{Eigenschaften von linearen Isometrien}
    Eine lineare Isometrie \( \abb{f}{V}{V} \)
    ist normal. Es gilt \( f^H = f^{-1} \).\\
    Sei \( K\in \set{\R, \C} \) der jeweilige 
    Grundkörper. Dann gilt
    \begin{enumerate}
        \item \( \spec_K(f) \subset \set{ x\in K \;\vert\; 
        \abs{x} = 1 } \).
        \item Sind \(\lambda, \mu\in\spec_K(f)\) 
        unterschiedliche Eigenwerte, dann ist 
        \[ E_\lambda(f) \bot E_\mu(f). \]
    \end{enumerate}
    Eine lineare Isometrie \(f\) eines unitären 
    Vektorraums \(V\) besitzt eine Orthonormalbasis 
    aus Eigenvektoren. Die Eigenwerte von \(f\) 
    haben die Norm \(1\).
\end{karte}

\begin{karte}{Diagonalisierung von Matrizen aus \(\mathrm{U}(n)\)}
    Ist \(A\in \mathrm{U}(n)\) eine unitäre Matrix, 
    so gibt es eine Matrix \( S \in \mathrm{U}(n) \) 
    und komplexe Zahlen \( \lambda_1, \ldots, \lambda_n \) 
    von Norm \(1\) so, dass 
    \[ S^H \cdot A \cdot S = S^{-1} \cdot A \cdot S 
    = \begin{pmatrix}
        \lambda_1 && 0 \\
        & \ddots & \\
        0 && \lambda_n
    \end{pmatrix}. \]
\end{karte}

\begin{karte}{Euklidische Normalform}
    Es sei \( \abb{f}{V}{V} \) eine lineare 
    Isometrie des euklidischen Vektorraums \(V\). 
    Dann besitzt \(V\) eine Orthonormalbasis \(B\)
    und es gibt Zahlen \(p, q, k \in \N_0\) und 
    Winkel \( \phi_1,\ldots, \phi_k \in (0,2\pi)
    \setminus \set{\pi} \) derart, dass 
    \[ M_{B,B}(f) = I_p \boxplus -I_q 
    \boxplus R(\phi_1) \boxplus \cdots \boxplus R(\phi_k). \]
    Dabei bezeichnet 
    \[ R(\phi) = \begin{pmatrix}
        \cos \phi & -\sin \phi \\
        \sin \phi & \cos \phi
    \end{pmatrix} \]
    die Matrix der Drehung um den Winkel \( \phi \).
\end{karte}

\begin{karte}{Lineare Unabhängigkeit über \(\C\)}
    Vektoren \( x_1\ldots, x_k \in \R^n \) sind 
    genau dann über \( \R \) linear unabhängig, 
    wenn sie über \(\C\) linear unabhängig sind 
    (als Vektoren in \( \R^n \subset \C^n \)).
\end{karte}

\begin{karte}{Basenunabhängigkeit in \(\R\) und \(\C\)}
    Eine \(\R\)-Basis von \( \ker L(A) \subset \R^n \) 
    ist zugleich eine \( \C \)-Basis von \\
    \( \ker L_\C (A)\subset \C^n \).
\end{karte}

\begin{karte}{Ent-Diagonalisierung der Rotationsmatrix}
    Sei \(A \in \mathrm{O}(n)\). Zu jedem Paar von 
    Eigenvektoren \( z,\overline{z} \) von \( L_\C(A) \), 
    zu nicht-reellen Eigenwerten \( \mu, \overline{\mu} \), 
    wobei 
    \[ \mu = \cos \alpha + i \cdot \sin \alpha, \]
    existiert ein \(2\)-dimensionaler \( L(A) \)-invarianter 
    \( \R \)-Untervektorraum \( W_z \subset \R^n \) und 
    eine ON-Basis \( B_z = (x_z, y_z) \) von \( W_z \) 
    so, dass 
    \[ \langle W_z \rangle_\C = \langle z, \overline{z}\rangle_\C \subset \C^n, \]
    \[ M_{B_z,B_z}(L(A)\vert_{W_z}) = R(\alpha)
    = \begin{pmatrix}
        \cos \alpha & -\sin \alpha \\
        \sin \alpha & \cos \alpha
    \end{pmatrix}. \]
\end{karte}

\begin{karte}{Spiegelung}
    Sei \( v\in V\setminus \set{0} \). Die 
    \textit{Spiegelung an der zu \(v\) orthogonalen Hyperebene} 
    ist die lineare Abbildung 
    \[ \abb{\sigma}{V}{V} \]
    \[ \sigma_v(x) := x - 2\frac{\scalarprod{x}{v}}{\scalarprod{v}{v}} \cdot v. \]
\end{karte}

\begin{karte}{Spiegelung zwischen zwei Vektoren}
    Seien \(v,w\) zwei verschiedene Vektoren 
    gleicher Länge \( \norm{v} = \norm{w} \). Sei 
    \( d = v-w \). Dann ist \( \sigma_d(w) = v \).
\end{karte}

\begin{karte}{Isometrien als Komposition von Spiegelungen}
    Jede lineare Isometrie eines \(n\)-dimensionalen 
    euklidischen Vektorraums ist eine Komposition von 
    höchstens \(n\) Spiegelungen.
\end{karte}

\begin{karte}{Eigenschaften der speziellen orthogonalen Gruppe}
    \begin{enumerate}
        \item Es gilt \( \mathrm{SO}(2) = \set{R(\phi) \;\vert\; 
        \phi \in [0,2\pi)} \). D. h. \( \mathrm{SO}(2) \) ist 
        die Gruppe der Rotationen in \( \R^2 \) um den Ursprung.
        \item Das Komplement \( \mathrm{O}(2) \setminus 
        \mathrm{SO}(2) \) besteht genau aus den Spiegelungen 
        von \(\R^2\).
        \item Für alle \( n\in\N \) gilt 
        \item \[ \mathrm{O}(n) = \mathrm{SO}(n) \cup 
        \set{S\cdot A \;\vert\; A \in \mathrm{SO}(n)}, \] 
        wobei \(S\) die Spiegelungsmatrix ist: 
        \[ S = \begin{pmatrix}
            -1 & 0 &\cdots & 0 \\
            0 & 1 && \vdots \\
            \vdots && \ddots & 0 \\
            0 & \cdots & 0 & 1
        \end{pmatrix} \in \mathrm{O}(n) \]
    \end{enumerate}
\end{karte}

\begin{karte}{\( \mathrm{SO}(3) \)}
    Die Gruppe \( \mathrm{SO}(3) \) besteht aus 
    Drehungen um Geraden durch den Ursprung. \\
    Genauer: Ist \(A \in \mathrm{SO}(3)\), so gibt 
    es \( B\in \mathrm{O}(3) \) und 
    \( \phi \in [0,2\pi) \) mit 
    \[ B^{-1} \cdot A \cdot B = 
    \begin{pmatrix}
        1 & 0 \\ 0 & R(\phi)
    \end{pmatrix} \in \mathrm{SO}(3). \]
\end{karte}

\begin{karte}{\( \mathrm{SO}(n) \) wegzusammenhängend}
    Der topologische Raum \( \mathrm{SO}(n) \) ist 
    wegzusammenhängend. D. h. für \( A,B\in \mathrm{SO}(n) \) 
    gibt es einen stetigen Weg \( \abb{\gamma}{[0,1]}{\mathrm{SO}(n)} \)
    mit \( \gamma(0) = A \) und \( \gamma(1) = B \).
\end{karte}

\begin{karte}{Gleich orientierte Basen}
    Zwei ON-Basen des \(\R^n\) sind \textit{gleich orientiert}, 
    wenn sie durch einen stetigen Weg auseinander hervorgehen.\\
    Zwei ON-Basen sind gleich orientiert, wenn die 
    Determinanten der zugehörigen Matrizen in \( \mathrm{O}(n) \) 
    das gleiche Vorzeichen haben.
\end{karte}

\begin{karte}{Affine Abbildung}
    Eine Abbildung \( \abb{f}{V}{W} \) zwischen Vektorräumen 
    heißt \textit{affin}, wenn es eine lineare Abbildung 
    \( \abb{g}{V}{W} \) und einen Vektor \( b \in W \) mit 
    \[ f(x) = g(x) + b \text{ für alle }x\in V \]
    gibt. D. h. eine affine Abbildung ist die Komposition 
    einer linearen Abbildung mit einer Translation.\\
    Weiter ist \(f\) ein \textit{affiner Isomorphismus} 
    oder eine \textit{Affinität}, wenn \(f\) eine affine 
    Umkehrabbildung besitzt.
\end{karte}

\begin{karte}{Isometrie zwischen metrischen Räumen}
    Eine bijektive Abbildung \( \abb{f}{X}{Y} \) 
    zwischen metrischen Räumen \( (X, d_X) \) und 
    \( (Y, d_Y) \) heißt \textit{Isometrie}, wenn 
    \[ d_Y(f(x), f(x')) = d_X(x,x') \] 
    für alle \( x,x' \in X \) erfüllt ist.
\end{karte}

\begin{karte}{Kongruenz}
    Eine Abbildung zwischen euklidischen 
    Vektorräumen ist eine \textit{Kongruenz}
    oder \textit{affine Isometrie}, wenn \(f\) 
    zugleich affin und Isometrie ist.\\
    Jede Isometrie eines euklidischen Vektorraums ist 
    eine affine Isometrie.
\end{karte}

\subsection*{Quadratische Formen}

\begin{karte}{Quadratische Form}
    Eine \textit{quadratische Form} auf einem 
    \(K\)-Vektorraum ist eine Abbildung 
    \[ \abb{q}{V}{K} \] 
    mit den folgenden beiden Eigenschaften 
    \begin{enumerate}
        \item \( q(\lambda v) = \lambda^2 q(v) \) für alle 
        \( \lambda \in K \) und \(v\in V\).
        \item Die Abbildung \( \abb{f}{V\times V}{K} \) 
        definiert eine Bilinearform durch 
        \[ f(v,w) = q(v+w) -q(v) -q(w). \]
    \end{enumerate}
\end{karte}

\begin{karte}{Bijektion zwischen quadratischen Formen und symmetrischen Bilinearformen}
    Sei \(K\) ein Körper mit \( 1_K + 1_K \neq 0_K \). Sei 
    \(V\) ein \( K \)-Vektorraum. Dann sind die folgenden 
    Konstruktionen zueinander inverse Bijektionen zwischen 
    symmetrischen Bilinearformen und quadratischen Formen 
    auf \(V\):
    \begin{enumerate}
        \item Einer symmetrischen Bilinearform 
        \( \abb{b}{V\times V}{K} \) ordnen wir 
        die quadratische Form 
        \[ q(v) = b(v,v) \]
        für alle \( v\in V \) zu.
        \item Einer quadratischen Form \( \abb{q}{V}{K} \) 
        ordnen wir die symmetrische Bilinearform 
        \[ f(v,w) = \frac{1}{2} (q(v+w) - q(v) - q(w)) \] 
        für alle \(v,w\in V\) zu.
    \end{enumerate}
\end{karte}

\begin{karte}{Quadratischer Raum}
    Einen \(K\)-Vektorraum \(V\) mit einer quadratischen 
    Form nennt man einen \textit{quadratischen Raum} 
    über \(K\).
\end{karte}

\begin{karte}{Isometrie quadratischer Räume}
    Eine Isometrie quadratischer Räume 
    \((V,q)\) und \((V',q')\) ist eine bijektive 
    lineare Abbildung \( \abb{f}{V}{V'} \) so, 
    dass 
    \begin{center}
        \begin{tikzcd}
            V \arrow{r}{f} \arrow{d}{q} & V' \arrow{d}{q'} \\
            K \arrow{r}{\id} & K
        \end{tikzcd}
    \end{center}
    kommutiert. Existiert eine Isometrie, 
    schreiben wir \( (V,q) \cong (V',q') \).
\end{karte}

\begin{karte}{Kongruente Matrizen}
    Zwei Matrizen \( A,B\in M_n(K) \) sind 
    \textit{kongruent}, wenn es ein \( S\in \GL_n(K) \) 
    gibt mit 
    \[ S^T \cdot A \cdot S = B. \]
\end{karte}

\begin{karte}{Abbildung zwischen Isometrieklassen und Kongruenzklassen}
    Die Abbildung 
    \begin{center}
        \begin{tikzcd}
            \set{ \text{Isometrieklassen \(n\)-dimensionaler quadratischer Räume über \(K\)} } \arrow{d} \\
            \set{ \text{Kongruenzklassen symmetrischer Matrizen in \(M_n(K)\)} },
        \end{tikzcd}
    \end{center}
    die \( [(V,b)] \) die Kongruenzklasse \( [M_B(b)] \) 
    der Matrixdarstellung der symmetrischen Bilinearform \(b\) 
    bezüglich irgendeiner Basis \(B\) von \(V\) zuordnet, 
    ist bijektiv.
\end{karte}

\begin{karte}{Abbildung zwischen Isomorphieklassen und Ähnlichkeitsklassen}
    Die Abbildung 
    \begin{center}
        \begin{tikzcd}
            \set{ \text{Isomorphieklassen \(n\)-dimensionaler Endoräume über \(K\)} } \arrow{d} \\
            \set{ \text{Ähnlichkeitsklassen in \(M_n(K)\)} },
        \end{tikzcd}
    \end{center}
    die \( [(V,g)] \) die Ähnlichkeitsklasse \( [M_{B,B}(g)] \) 
    der Matrixdarstellung von \(g\) bezüglich irgendeiner 
    Basis \(B\) von \(V\) zuordnet, ist bijektiv.
\end{karte}

\begin{karte}{Signatur von Bilinearformen}
    Sei \(b\) eine symmetrische Bilinearform auf 
    einem endlich-dimensionalen \(\R\)-Vektorraum \(V\). 
    Es seien 
    \begin{align*}
        n_+ &:= \max\set{ \dim_\R U \;\vert\; U \subset V \text{ Unter-VR und } b\vert_{U\times U} \text{ positiv definit} }, \\
        n_- &:= \max\set{ \dim_\R U \;\vert\; U \subset V \text{ Unter-VR und } b\vert_{U\times U} \text{ negativ definit} }, \\
        n_0 &:= \max\set{ \dim_\R U \;\vert\; U \subset V \text{ Unter-VR und } b\vert_{U\times U} = 0 }.
    \end{align*}
    Dann bezeichnet man die Differenz \( n_+ - n_- \) 
    als \textit{Index} von \(b\) und das Tripel \( (n_+, n_-, n_0) \)
    als \textit{Signatur} von \(b\).
\end{karte}

\begin{karte}{Trägheitssatz von Sylvester}
    Sei \( V \) ein endlich-dimensionaler \( \R \)-Vektorraum 
    mit symmetrischer Bilinearform \(b\). Sei \(A = M_B(b) \in M_n(\R)\) 
    die darstellende Matrix von \(b\) bezüglich einer Basis 
    \(B\). Sei \( (n_+,n_-,n_0) \) die Signatur von \(b\). 
    Dann sind 
    \[ n_+ = \sum_{\lambda\in \spec_\R(A)\cap(0,\infty)} \dim_\R E_\lambda(A), \]
    \[ n_- = \sum_{\lambda\in \spec_\R(A)\cap(-\infty,0)} \dim_\R E_\lambda(A), \]
    \[ n_0 = \dim_\R E_0(A) = \dim_\R \ker A. \]
    Insbesondere ist \( n = n_+ + n_- + n_0 \).
\end{karte}

\begin{karte}{Rang einer Bilinearform}
    Der \textit{Rang} einer symmetrischen Bilinearform \(b\) 
    auf einem \( K \)-Vektorraum ist der Rang einer 
    darstellenden Matrix \( M_B(b) \) für irgendeine 
    Basis \(B\) von \(V\).
\end{karte}

\begin{karte}{Kongruenz zu Sylvesterform}
    Sei \(b\) eine symmetrische Bilinearform auf einem 
    \(n\)-dimensionalen \\
    \(\R\)-Vektorraum \(V\) mit 
    Signaturtripel \( (n_+, n_-, n_0) \). Dann gibt 
    es eine Basis \(C\) von \(V\) so, dass 
    \[ M_C(b) = \begin{pmatrix}
        I_{n_+} & 0 & 0 \\
        0 & -I_{n_-} & 0 \\
        0 & 0 & 0
    \end{pmatrix} \in M_n(\R). \]
    Insbesondere sind zwei quadratische Räume über \(\R\) 
    genau dann isometrisch, wenn ihre Dimensionen und 
    Signaturen übereinstimmen.
\end{karte}

\begin{karte}{Kongruenz zu Diagonalmatrizen}
    Es sei \(K\) ein Körper mit \( 1_K + 1_K \neq 0_K \). 
    Sei \(A \in M_n(K)\) eine symmetrische Matrix. Dann 
    gibt es \( S \in \GL_n(K) \) so, dass \(S^T \cdot A \cdot S\) 
    eine Diagonalmatrix ist.
\end{karte}

\begin{karte}{Isometrie zwichen quadratischen Räumen über \(\C\)}
    Sei \(b\) eine symmetrische Bilinearform auf einem 
    \(n\)-dimensionalen \\
    \(\C\)-Vektorraum. Dann gibt es eine 
    Basis \(C\) von \(V\) so, dass 
    \[ M_C(b) = \begin{pmatrix}
        I_k & 0 \\
        0 & 0
    \end{pmatrix} \in M_n(\R). \]
    Insbesondere sind zwei quadratische Räume über \(\C\) 
    genau dann isometrisch, wenn ihre Dimensionen und Ränge 
    übereinstimmen.
\end{karte}

\subsection*{Quadriken}

\begin{karte}{Quadrik}
    Sei \( q \) eine nicht-verschwindende quadratische 
    Form auf \( \R^n, b\in \R^n \) und \( c\in \R \). 
    Eine \textit{Quadrik} in \( \R^n \) ist eine Menge 
    der Form 
    \[ Q = \set{x \in \R^n \;\vert\; q(x) + 2\scalarprod{b}{x}_2 = c}. \]
\end{karte}

\begin{karte}{Kongruente Quadriken}
    Zwei Quadriken \(Q_1\) und \(Q_2\) in 
    \(\R^n\) sind \textit{kongruent}, wenn es 
    eine Kongruenz \( \abb{f}{\R^n}{\R^n} \) 
    mit \( f(Q_1) = Q_2 \) gibt.
\end{karte}

\begin{karte}{Hauptachsentransformation}
    Eine Quadrik des \( \R^n \) 
    \[ Q = \set{ x\in \R^n \;\vert\; \scalarprod{x}{Ax}_2 
    + 2 \cdot \scalarprod{x}{b}_2 + c = 0 } \]
    ist kongruent zu einer Quadrik mit einer Gleichung der 
    folgenden Form (\textit{metrische Normalform}):
    \begin{align*}
        \sum_{i=1}^r \frac{x_i^2}{a_i^2} - \sum_{j=r+1}^{r+s} \frac{x_j^2}{a_j^2} &= 0. \\
        \sum_{i=1}^r \frac{x_i^2}{a_i^2} - \sum_{j=r+1}^{r+s} \frac{x_j^2}{a_j^2} &= 1. \\
        \sum_{i=1}^r \frac{x_i^2}{a_i^2} - \sum_{j=r+1}^{r+s} \frac{x_j^2}{a_j^2} &= x_{r+s+1}.
    \end{align*}
    Dabei gilt \( a_i > 0 \) für \( i \in \set{1, \ldots, r+s} \).
\end{karte}

\begin{karte}{Euklidische Normalform im \(\R^2\)}
    Eine Quadrik im \( \R^2 \) ist kongruent zu 
    eine der durch folgende Gleichungen 
    beschriebenen Quadriken (dabei ist \(a_1, a_2 \neq 0\)):\\
    {\renewcommand{\arraystretch}{1.5}
    \begin{center}
        \begin{tabular}{|r|l|}
            \hline
            \( \frac{x_1^2}{a_1^2} + \frac{x_2^2}{a_2^2} = 0 \) & Punkt \\
            \hline
            \( \frac{x_1^2}{a_1^2} - \frac{x_2^2}{a_2^2} = 0 \) & schneidendes Geradenpaar \\
            \hline
            \( \frac{x_1^2}{a_1^2} = 0 \) & Doppelgerade \\
            \hline
            \( \frac{x_1^2}{a_1^2} - \frac{x_2^2}{a_2^2} = 1 \) & Hyperbel \\
            \hline 
            \( \frac{x_1^2}{a_1^2} + \frac{x_2^2}{a_2^2} = 1 \) & Ellipse \\
            \hline 
            \( - \frac{x_1^2}{a_1^2} = 1 \) oder \( - \frac{x_1^2}{a_1^2} - \frac{x_2^2}{a_2^2} = 1 \) & leere Menge \\
            \hline 
            \( \frac{x_1^2}{a_1^2} = 1 \) & paralleles Geradenpaar \\
            \hline 
            \( \frac{x_1^2}{a_1^2} = 2x_2 \) & Parabel \\
            \hline
        \end{tabular}
    \end{center}}
\end{karte}

\begin{karte}{Euklidische Normalform im \( \R^3 \) Teil 1}
    Eine von \( \emptyset \), Punkt, Gerade und Ebene verschiedene Quadrik des 
    \( \R^3 \) ist kongruent zu einer der folgenden Quadriken: 
    {\renewcommand{\arraystretch}{1.5}
    \begin{center}
    \begin{tabular}{|r|l|}
        \hline
        \( \frac{x_1^2}{a_1^2} + \frac{x_2^2}{a_2^2} - \frac{x_3^2}{a_3^2} = 0 \) & Kegel \\
        \( \frac{x_1^2}{a_1^2} - \frac{x_2^2}{a_2^2} = 0 \) & Paar sich schneidender Ebenen \\
        \hline 
        \( \frac{x_1^2}{a_1^2} + \frac{x_2^2}{a_2^2} + \frac{x_3^2}{a_3^2} = 1 \) & Ellipsoid \\
        \( \frac{x_1^2}{a_1^2} + \frac{x_2^2}{a_2^2} - \frac{x_3^2}{a_3^2} = 1 \) & Einschaliges Hyperboloid \\
        \( \frac{x_1^2}{a_1^2} - \frac{x_2^2}{a_2^2} - \frac{x_3^2}{a_3^2} = 1 \) & Zweischaliges Hyperboloid \\
        \hline
    \end{tabular}
    \end{center}}
\end{karte}

\begin{karte}{Euklidische Normalform im \( \R^3 \) Teil 2}
    {\renewcommand{\arraystretch}{1.5}
    \begin{center}
    \begin{tabular}{|r|l|}
        \hline
        \( \frac{x_1^2}{a_1^2} + \frac{x_2^2}{a_2^2} = 1 \) & Elliptischer Zylinder \\
        \( \frac{x_1^2}{a_1^2} - \frac{x_2^2}{a_2^2} = 1 \) & Hyperbolischer Zylinder \\
        \( \frac{x_1^2}{a_1^2} = 1 \) & Paar paralleler Ebenen \\
        \hline 
        \( \frac{x_1^2}{a_1^2} + \frac{x_2^2}{a_2^2} = 2x_3 \) & Elliptisches Paraboloid \\
        \( \frac{x_1^2}{a_1^2} - \frac{x_2^2}{a_2^2} = 2x_3 \) & Hyperbolisches Paraboloid \\
        \( \frac{x_1^2}{a_1^2} = 2x_3 \) & Parabolischer Zylinder \\
        \hline
    \end{tabular}
    \end{center}}
\end{karte}

\end{document}